\documentclass [a4paper,11pt]{article}
\usepackage{enumitem}
\usepackage[utf8]{inputenc}
\usepackage[T1]{fontenc}
%\usepackage{cite} 
%\usepackage[sort&compress,square,numbers]{natbib}
\usepackage[backend=biber,sorting=none]{biblatex}
\addbibresource{references.bib}
\usepackage{color}
\setlength{\itemsep}{0pt plus4pt minus2pt}
\renewcommand{\labelitemi}{\Alph{enumi}}
\setlength{\headheight}{0mm} \setlength{\topskip}{0mm}
\setlength{\textheight}{24cm} \setlength{\topmargin}{-15mm}
\setlength{\textwidth}{16cm}\setlength{\oddsidemargin}{0cm}\setlength{\evensidemargin}{0cm}
\setlength{\parindent}{0mm} \setlength{\parskip}{2mm}
\usepackage{url}
\usepackage{amsthm}
\usepackage{amsmath}
\usepackage{amssymb}
\usepackage{eurosym}
\usepackage{enumitem}
\usepackage{graphicx}
\usepackage[colorinlistoftodos]{todonotes}
\usepackage{booktabs}
\usepackage[version=4]{mhchem}
\usepackage[subrefformat=parens,labelformat=parens]{subfig}
\usepackage{comment}
\usepackage{bm}
\graphicspath{ {./}{./figs/}{./figs/laprint_figs/} }



\definecolor{SteelBlue}  {rgb}{0.275,0.51,0.706}  

\usepackage{xr}
\externaldocument[m-]{main}

\newtheoremstyle{blue}% name
{10pt}%      Space above
{3pt}%      Space below
{\color{SteelBlue}}%         Body font
{}%         Indent amount (empty = no indent, \parindent = para indent)
{\bfseries}% Thm head font
{:}%        Punctuation after thm head
{.5em}%     Space after thm head: " " = normal interword space;
%       \newline = linebreak
{}%         Thm head spec (can be left empty, meaning `normal')

\theoremstyle{blue}
\newtheorem{question}{Comment}[section]

\definecolor{green-red}  {rgb}{0,0.,1}
\newtheoremstyle{blue}% name
{10pt}%      Space above
{3pt}%      Space below
{\color{green-red}}%         Body font
{}%         Indent amount (empty = no indent, \parindent = para indent)
{\bfseries}% Thm head font
{:}%        Punctuation after thm head
{.5em}%     Space after thm head: " " = normal interword space;
%       \newline = linebreak
{}%         Thm head spec (can be left empty, meaning `normal')

\theoremstyle{blue}
\newtheorem{general}{General Comment}[section]

\newtheoremstyle{note}% name
{3pt}%      Space above
{3pt}%      Space below
{}%         Body font
{}%         Indent amount (empty = no indent, \parindent = para indent)
{\bfseries}% Thm head font
{:}%        Punctuation after thm head
{.5em}%     Space after thm head: " " = normal interword space;
%       \newline = linebreak
{}%         Thm head spec (can be left empty, meaning `normal')

\theoremstyle{note}
\newtheorem{answer}{Response}[section]

\newtheoremstyle{note}% name
{3pt}%      Space above
{10pt}%      Space below
{}%         Body font
{}%         Indent amount (empty = no indent, \parindent = para indent)
{\bfseries}% Thm head font
{:}%        Punctuation after thm head
{.5em}%     Space after thm head: " " = normal interword space;
%       \newline = linebreak
{}%         Thm head spec (can be left empty, meaning `normal')

\theoremstyle{note}
\newtheorem{general_answer}{General Response}[section]

\newcommand{\todopb}[2][]
{\todo[backgroundcolor=blue!25, #1]{#2}}

\usepackage[normalem]{ulem}
\newcommand{\mkom}[1]{{\color{purple} #1}}
\newcommand{\mnew}[1]{{\color{Tan} #1}}
\newcommand{\mpop}[2]{{\color{orange}\sout{#1}\color{purple}{#2}}}
\newcommand{\mrev}[2]{{\color{red}\footnote{#1} #2 }}

\usepackage[nomargin,inline,marginclue,draft]{fixme}
 \fxsetup{theme=color, mode=multiuser} 
 \FXRegisterAuthor{pb}{apb}{\color{red}PB} 
 \FXRegisterAuthor{mp}{amp}{\color{red}mp}

\usepackage{xr}
\externaldocument[M-]{detcoupling}

\begin{document}
	
	\textbf{Detailed Response to Reviewers}\\[10pt]
	
%	{\large \textbf{Reply to the reviewers}}
%	This document contains the responses to the reviewers' questions.
%	In addition to the revised manuscript, we have added a marker revised manuscript that includes color-highlighted changes marked with a reference to the reviewers' questions.
	
%	The analysis of the mutual dependence of $(N,L)$ and the dependence on the parameters $k_1$ and $k_2$ are presented as Supplementary material.
%	We have also moved the original Appendix~A as a part of the Supplementary material.
	
	\section{Reviewer \#1:}

%{\color{blue}Section I)}

\begin{question}
Correct small typographical issues (e.g., replace "it's" with "its" in formal writing).
\end{question}

\begin{answer}

Thank you. This is now corrected.

\end{answer}

%{\color{blue}Section II)}

\begin{question}

Improve Figure 1's caption to clearly distinguish ensemble vs. variational trees.

\end{question}

\begin{answer}

The caption has been improved as per your suggestion.

\end{answer}

\begin{question}

Consider summarizing model characteristics in a comparative table in the Related Work section.

\end{question}

\begin{answer}

We have now included a comparative table in the Related Work section clearly summarising and comparing the key characteristics of the VSPYCT method and the other predictive clustering methods in the literature.

\end{answer}

\begin{question}

Clarify "structured output prediction" early for readers not familiar with the term.

\end{question}

\begin{answer}

We clarified the term Structured Output Prediction (SOP) explicitly in the Introduction section by providing a concise definition for readers who may not be familiar with this concept.

\end{answer}

\begin{question}

Include explanation of $\Omega_{obs} = \Omega (X,Y,w,b)/2$ mentioned in Algorithm1 and why this choice is made.

\end{question}

\begin{answer}

We have clarified this in the manuscript (inside Algorithm 1), explicitly stating that the factor of 1/2 in $\Omega_{obs} = \Omega (X,Y,w,b)/2$ is introduced to ensure that it is consistently lower than the current impurity $\Omega (X,Y,w,b)$. This guarantees that the optimisation is biased toward impurity reduction at each step, thereby encouraging effective split selection. The choice helps to drive the learning dynamics in the right direction by maintaining a useful margin between the observed and expected impurities.

\end{answer}

\begin{question}

Although the paper acknowledges the added complexity of variational inference compared to deterministic tree splits, it lacks empirical evidence regarding training time
or scalability.

\end{question}

\begin{answer}

Thank you for highlighting this important aspect. While we have not provided empirical training-time measurements due to their high dependence on specific implementations and hardware, we have explicitly derived and clearly presented the theoretical computational complexity of our proposed VSPYCT model compared to the deterministic SPYCT. The added complexity due to the variational inference framework is detailed explicitly in the manuscript. To clarify the limitations, we have now clearly stated in the manuscript that systematic empirical evaluations of training time and scalability were outside of the scope of this study, due to the above-mentioned reasons.

\end{answer}

\begin{question}

The model captures parameter uncertainty well, but structural uncertainty (e.g., tree topology) is briefly mentioned as future work.

\end{question}

\begin{answer}

Thank you for this important suggestion. The Discussion section now highlights structural uncertainty as a significant area for future research. Specifically, we mentioned potential strategies such as incorporating structural priors or employing ensemble-like approaches within the variational framework.

\end{answer}


\begin{question}

There is no mention of code availability, parameter tuning strategies, implementation frameworks or datasets . Consider including a statement on code or data
availability to enhance reproducibility.

\end{question}

\begin{answer}

We have now included a clear statement in the manuscript with a link to our code implementation.

\end{answer}

\begin{question}

While feature importance and visualisations are well described, a brief practical commentary on how domain experts might interact with or interpret the VSPYCT outputs
would be valuable.

\end{question}

\begin{answer}

We have explicitly added a brief practical commentary in the Interpretability subsection, clearly outlining how domain experts (e.g., clinicians, employment analysts, or material scientists) might effectively interact with and interpret VSPYCT outputs for informed decision-making in their respective fields.

\end{answer}


	\section{Reviewer \#2:}
	
	
\begin{question}

The novelty of the work should be clearly emphasized in the last paragraph of the introduction.

\end{question}

\begin{answer}

Thank you for this important remark. We have now explicitly emphasised the novelty of our approach in the final paragraph of the Introduction, clearly highlighting the integration of Bayesian inference with oblique splits, the direct quantification of uncertainty, and the unique capability to maintain interpretability while achieving competitive predictive performance.

\end{answer}

\begin{question}

Too many parameters have been examined in the study, but few results are presented in the Conclusion section. Please add a few points in the Conclusion section.

\end{question}

\begin{answer}

We have enhanced the Conclusion section with specific results summarizing the predictive performance of VSPYCT compared to benchmark methods. Additionally, we explicitly discussed key findings, limitations, and recommendations for future research clearly.

\end{answer}

\begin{question}

All parameters and symbols mentioned in this manuscript must be defined within the manuscript or in a separate list of nomenclature.

\end{question}

\begin{answer}

Thank you for this. All parameters and symbols used throughout the manuscript have now been explicitly defined at the point of introduction within the main text. We have ensured that no undefined notation remains, making the manuscript self-contained without the need for a separate nomenclature table.

\end{answer}

\begin{question}

The introduction is not well organized. The research gap and novelty of the research are missing. The author should clearly explain what has done previously and what is the research gap. The authors did not explain the cited articles properly.

\end{question}

\begin{answer}

Thank you for highlighting this issue. We reorganized the Introduction section by explicitly identifying the research gap—particularly emphasizing the lack of integrated uncertainty quantification and interpretability in previous studies. We have also clarified the context and relevance of the cited literature.

\end{answer}

\begin{question}

For the validity of the results, the study needs to be validated with previous studies graphically.

\end{question}

\begin{answer}

Thank you for highlighting this important point.
To clarify, our experimental design already includes comparisons with state-of-the-art models from previous literature, such as standard SPYCT, ensembles of SPYCTs, OPCT, and ensembles of PCTs.
These baseline models have been extensively evaluated and validated in previous published studies (Stepišnik and Kocev, 2021; Kocev et al., 2013; Stepisnik et al., 2020), and we explicitly included them in our empirical analyses. The ranking based on the predictive performance is given in Figure 2.

\end{answer}

\begin{question}

What is the novelty of this research?

\end{question}

\begin{answer}

The novelty of our research lies in the unique integration of variational Bayesian inference with oblique predictive clustering trees. This allows us to directly quantify uncertainty within a single interpretable tree model, achieving predictive performance comparable or superior to state-of-the-art ensemble methods, which has not been explored in previous studies.

\end{answer}

\begin{question}

The result and discussion section should be improved from the physical point of view.

\end{question}

\begin{answer}

We have introduced a new subsection "Practical Interpretation of Results" in the Results and Discussion section. This clearly addresses the physical and practical significance of our findings, explicitly linking the predictive outcomes to real-world contexts such as material properties, medical diagnoses, and employment data.

\end{answer}

\begin{question}

How were similarity parameters calculated?

\end{question}

\begin{answer}

We have explicitly clarified in the Methodology section how similarity parameters, specifically feature importance, are calculated. These parameters are computed based on the ratio of the expected values of the weights to their variances, which integrates both the magnitude and uncertainty of each feature's contribution to the predictions.

\end{answer}

\begin{question}

Include justifications for the choice of specific parameters and models.

\end{question}

\begin{answer}

We have clearly included justifications for our chosen parameters and models within the Methodology section. Specifically, we explained our choice of Gaussian priors for computational tractability, the selection of the ELBO optimisation approach, and parameters used in Monte Carlo sampling based on trade-offs between predictive accuracy and computational efficiency.

\end{answer}

\begin{question}

It is recommended to noticeably highlight the advantages, disadvantages, and limitations of the proposed approaches.

\end{question}

\begin{answer}

We have clearly added an explicit and structured discussion of the advantages, disadvantages, and limitations of the VSPYCT model within the Discussion subsection. Advantages include interpretability and uncertainty quantification, while disadvantages primarily involve computational complexity and potential scalability concerns.

\end{answer}

\begin{question}

It is necessary to incorporate further analysis and discussion comparing the current findings with previous publications. There must be an explanation for any differences between the present results and the data that has been published.

\end{question}

\begin{answer}



\end{answer}

\begin{question}

It is recommended to add a table representing all the investigated cases in the manuscript; please show the values/range of each tested parameter.

\end{question}

\begin{answer}

We have now added a comprehensive table under the "Experimental Setting" section, summarising all parameters investigated, their ranges, and their specific relevance. Furthermore, we explicitly stated the final selected parameter values used throughout the experiments clearly below the table.

\end{answer}

\begin{question}

The conclusion section should be better organized and it is recommended to include quantitative results. More specific aspects regarding the limitations, recommendations, and future works can be stated in the conclusion section. It is recommended to include qualitative and quantitative results.

\end{question}

\begin{answer}

We have reorganised and enhanced our Conclusion section to explicitly include both qualitative and quantitative summaries of our key findings.

\end{answer}

\begin{question}

Ensure that all sources and data are properly cited throughout the manuscript. For instance, the following papers could be helpful in improving the literature

\end{question}

\begin{answer}

Thank you for this suggestion. We have now cited the recommended references within our manuscript.

\end{answer}
	
\end{document}
